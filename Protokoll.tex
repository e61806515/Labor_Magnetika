\documentclass[a4paper,twoside,12pt,DIV=13,BCOR=5mm,numbers=noenddot,cleardoublepage=empty]{scrbook}
\input{format-LU} 
\usepackage{ucs} 	
\usepackage{float}		% File mit den ben�tigten Packeten, den Formatanweisungen und den Befehlsdefinitionen
\begin{document}
\newcommand\todo[1]{\textcolor{red}{#1}}
%=============================================================================================
% Titelblatt und Inhaltsverzeichnis
%=============================================================================================
\renewcommand{\baselinestretch}{1.25}
\newcommand{\StudentA}{Marton Harsch}
\newcommand{\MatrNrA}{12123680}
\newcommand{\StudentB}{Michael Malburg}
\newcommand{\MatrNrB}{61806515}
\newcommand{\StudentC}{Jonathan Gamperl}
\newcommand{\MatrNrC}{12302766}

\newcommand{\LUDatum}{22.4.2024}
\newcommand{\LUGruppe}{}
\newcommand{\LUBetreuer}{}

\large
\includepdf[fitpaper=true,
						picturecommand*={\unitlength1cm 
						\put(7.3,7.7){\StudentA} \put(14.1,7.7){\MatrNrA}
            \put(7.3,7.0){\StudentB} \put(14.1,7.0){\MatrNrB}
            \put(7.3,6.3){\StudentC} \put(14.1,6.3){\MatrNrC}
						\put(7.3,5.1){\LUDatum} 
            \put(7.3,4.4){\LUGruppe} 
            \put(7.3,3.7){\LUBetreuer} 
}]
{pictures/DeckblattLUDie}     %file name of title page


%===============================================================================
% Text
%===============================================================================

\cleardoublepage
\setcounter{tocdepth}{3}

\setcounter{page}{0}
\renewcommand{\thepage}{\roman{page}}
\tableofcontents \cleardoublepage

\setcounter{page}{1}
\renewcommand{\thepage}{\arabic{page}}
\setcounter{chapter}{0}


%===============================================================================

\newpage
\chapter{Zusammenfassung}
Die Zusammenfassung
\newpage
\chapter{Ringkerntrafo}
\section{Aufnahme einer Hystereseschleife}

\section{Aufnahme der Permeabilitätskurve}


\section{Hystereseschleife und Entmagnetisierung}


\section{Die Neukurve}

\section{Fazit}

\chapter{Übungen am Elektromagneten}
\section{Metallscheiben im Magnetfeld}

\section{Eisenblech im Magnetfeld}

\section{Diamagnetische Stoffe}

\section{Messung des Magnetfeldes}

\section{Lorentzkraft}

´\section{Fazit}



\subsection{Messung} 



\end{document}


